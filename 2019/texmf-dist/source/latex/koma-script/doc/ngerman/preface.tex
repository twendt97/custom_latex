% ======================================================================
% preface.tex
% Copyright (c) Markus Kohm, 2008-2019
%
% This file is part of the LaTeX2e KOMA-Script bundle.
%
% This work may be distributed and/or modified under the conditions of
% the LaTeX Project Public License, version 1.3c of the license.
% The latest version of this license is in
%   http://www.latex-project.org/lppl.txt
% and version 1.3c or later is part of all distributions of LaTeX
% version 2005/12/01 or later and of this work.
%
% This work has the LPPL maintenance status "author-maintained".
%
% The Current Maintainer and author of this work is Markus Kohm.
%
% This work consists of all files listed in manifest.txt.
% ----------------------------------------------------------------------
% preface.tex
% Copyright (c) Markus Kohm, 2008-2019
%
% Dieses Werk darf nach den Bedingungen der LaTeX Project Public Lizenz,
% Version 1.3c, verteilt und/oder veraendert werden.
% Die neuste Version dieser Lizenz ist
%   http://www.latex-project.org/lppl.txt
% und Version 1.3c ist Teil aller Verteilungen von LaTeX
% Version 2005/12/01 oder spaeter und dieses Werks.
%
% Dieses Werk hat den LPPL-Verwaltungs-Status "author-maintained"
% (allein durch den Autor verwaltet).
%
% Der Aktuelle Verwalter und Autor dieses Werkes ist Markus Kohm.
%
% Dieses Werk besteht aus den in manifest.txt aufgefuehrten Dateien.
% ======================================================================

\KOMAProvidesFile{typearea.tex}
                 [$Date: 2019-12-19 10:20:31 +0100 (Thu, 19 Dec 2019) $
                  Vorwort zu Version 3.25]

\addchap{Vorwort zu \KOMAScript~3.28}

Die Anleitung zu \KOMAScript~3.28 profitiert wieder einmal davon, dass nahezu
zeitgleich mit dieser Version auch eine überarbeitete Neuauflage der
Print-Ausgabe \cite{book:komascript} und der eBook-Ausgabe
\cite{ebook:komascript} erscheinen wird. Das führte zu vielen Verbesserungen,
die sich auch auf die freie Anleitung auswirken.

In \KOMAScript~3.28 gibt es darüber hinaus einige tiefgreifende
Änderungen. Teilweise wurde dabei auch auf Kompatibilität zu früheren
Versionen verzichtet. Damit wurde einer Empfehlung aus den Reihen von
\emph{The \LaTeX{} Project Team} bezüglich \Macro{if\dots}-Anweisungen
entsprochen. Wer solche verwendet, sollte die Anweisungen also in der
Anleitung erneut nachschlagen.

Nicht nur zur Anleitung erfahre ich inzwischen eher wenig Kritik. Ich schließe
daraus, dass \KOMAScript{} mittlerweile alles bietet, was Anwender sich
wünschen. Gleichzeitig hat das Projekt nicht erst mit der vorliegenden Version
einen Umfang erreicht, der es fast unmöglich macht, dass eine einzige Person
\begin{itemize}
\item die Suche nach und Beseitigung von Fehlern,
\item die Entwicklung und Realisierung neuer Funktionen,
\item die Beobachtung von Veränderungen bei anderen Paketen und dem
  \LaTeX-Kern im Hinblick auf Auswirkungen auf \KOMAScript,
\item die rasche Reaktion auf solche Veränderungen,
\item die Pflege der Anleitung in zwei Sprachen,
\item Hilfestellung für Anfänger weit über die Funktionen von \KOMAScript{}
  hinaus bis hin zur grundlegenden Bedienung eines Computers,
\item Hilfe bei der Umsetzung trickreicher Lösungen für fortgeschrittene
  Anwender und Experten,
\item Moderation und Mitwirkung bei der Pflege eines Forums für alle Arten der
  Hilfestellung rund um \KOMAScript
\end{itemize}
leisten kann. Während mir persönlich die Entwicklung neuer Funktionen am
meisten Spaß bereitet, halte ich die Fehlerbehebung in vorhandenen Funktionen,
Kompatibilität mit neuen \LaTeX-Kernel-Versionen, aber vor allem die Anleitung
von Anwendern für die wichtigsten Aufgaben. Daher werde ich mich künftig auf
diese Bereiche konzentrieren und neue Funktionen nur noch in Ausnahmefällen in
\KOMAScript{} integrieren. Bereits in \KOMAScript~3.28 wurden daher einige
experimentelle Funktionen und Pakete wieder entfernt. In zukünftigen Versionen
soll dies fortgesetzt werden.

Damit schwindet natürlich auch der Aufwand für die Dokumentation immer neuer
Funktionen. Leser dieser freien Bildschirm-Version der Anleitung müssen aber
auch weiterhin mit gewissen Einschränkungen leben. So sind einige
Informationen -- hauptsächliche solche für fortgeschrittene Anwender oder die
dazu geeignet sind, aus einem Anwender einen fortgeschrittenen Anwender zu
machen -- der Buchfassung vorbehalten. Das führt auch dazu, dass weiterhin
einige Links in dieser Anleitung lediglich zu einer Seite führen, auf der
genau diese Tatsache erwähnt ist. Darüber hinaus ist die freie Version nur
eingeschränkt zum Ausdruck geeignet. Der Fokus liegt vielmehr auf der
Verwendung am Bildschirm parallel zur Arbeit an einem Dokument. Sie hat auch
weiterhin keinen optimierten Umbruch, sondern ist quasi ein erster Entwurf,
bei dem Absatz- und Seitenumbruch in einigen Fällen durchaus dürftig
sind. Entsprechende Optimierungen bleiben den Buchausgaben vorbehalten.

Ganz besonders gedankt sei an dieser Stelle Elke Schubert. Elke ist seit
Jahren eine unverzichtbare Helferin. Sie nimmt nahezu jede Änderung an
\KOMAScript{} kritisch unter die Lupe. Sie meldet Probleme, die in
irgendwelchen Internetforen auftauchen, wenn sie vermuten muss, dass der
jeweilige Fragesteller dies wiedereinmal versäumt. Sie liest die Anleitung und
den Buchentwurf viele Dutzend Mal, probiert dabei jedes Beispiel aus und
diskutiert mit mir über sprachliche Grauzonen. Sie baut mich wieder auf, wenn
ich allzu oft aus unterschiedlichsten Gründen frustriert bin. Und sie
erträgt auch meine weniger liebenswerten Schrullen mit sehr viel Geduld.

Der größte Dank aber geht an meine Familie und allen voran meine Frau. Sie
federn all meine unschönen Erfahrungen im Internet ab. Ebenso erdulden sie
seit teilweise mehr als 25~Jahren, wenn ich wieder einmal nicht ansprechbar
bin, weil ich ganz und gar in \KOMAScript{} oder irgendwelche \LaTeX-Probleme
vertieft bin. Dass ich es mir leisten kann, überhaupt geradezu wahnsinnig viel
Zeit in ein derartiges Projekt zu investieren, ist allein meiner Frau zu
verdanken.

\bigskip\noindent
Markus Kohm, Neckarhausen im Dezembernebel 2019
\endinput

%%% Local Variables: 
%%% mode: latex
%%% coding: utf-8
%%% TeX-master: "../guide.tex"
%%% End: 

